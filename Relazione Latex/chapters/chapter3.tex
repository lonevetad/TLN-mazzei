\chapter{Conclusioni}
Il progetto proposto si è rivelato molto interessante, in quanto ha permesso un approfondimento pratico di quanto appreso durante la prima parte del corso di TLN. Potendo infatti applicare i vari concetti ha portato a capire meglio le fasi di elaborazione di una frase e tutte le parti correlate all'elaborazione stessa.\\
Il principale obiettivo è stato quello di rendere il programma abbastanza generale, non soffermandosi solo sulla singolarità della traduzione delle tre frasi. A questo proposito si è voluto testare il programma anche su altre frasi di vario tipo: si è notato che esiste una grande quantità di dipendenze tra le parole. Anche da questo si capisce quanto possa essere complessa l’implementazione di un traduttore transfer-sintattico. Infatti il soggetto sottinteso in alcune frasi italiane ci rende difficoltosa la traduzione, in quanto tale regola non esiste nella lingua inglese oppure altri aspetti come l'uso delle forme contratte e l'ordine diverso delle parole in una frase. 
