\chapter*{Introduzione} 
\addcontentsline{toc}{chapter}{Requirements}

Per il progetto proposto si chiedeva di realizzare un Traduttore Transfer sintattico da Italiano ad Inglese.\\
Quindi si è sviluppato un programma che, prendendo in input una frase in italiano, fornisse in output la corrispondente in inglese. Questo avviene mediante l'utilizzo di due risorse principali:
\begin{itemize}
	\item \textbf{Tint;}
	\item \textbf{SimpleNLG.}
\end{itemize}
Attraverso le componenti appena citate si sono potuti applicare i concetti visti durante il corso e, di conseguenza, "toccare con mano" il processo di elaborazione di una frase o di un testo, con tutte le sfaccettature che ne comporta.



 